%----------------------------------------------------------
\chapter{Тестирование и отладка}\label{chap4_soft_testing}
%----------------------------------------------------------
\section{Экспорт графа в aDOT файл}
Для тестировании основная функционал и экспорта в файл с помощью доступных инструментов сделаем граф показанный на рисунке(\ref{g1}).

В результате должен получиться файл с описание графовой модели в формате aDOT представленным на листинге (\ref{l1}).
\begin{lstlisting}[label=l1, caption={\textit{Полученное описание графовой модели в формате aDOT}}]
digraph TEST
{
// Parallelism
s1 [parallelism=threading]
s2 [parallelism=threading]
// Function
f1 [module=f1_module, entry_func=f1_function]
// Predicates
p1 [module=p1_module, entry_func=p1_function]
// Transition
edge_1 [predicate=p1, function=f1]
// Graph model
__BEGIN__ -> s1
s1 => s2 [morphism=edge_1]
s1 => s3 [morphism=edge_1]
s2 => s4 [morphism=edge_1]
s2 => s5 [morphism=edge_1]
s3 -> s7 [morphism=edge_1]
s4 -> s6 [morphism=edge_1]
s5 -> s6 [morphism=edge_1]
s6 -> s9 [morphism=edge_1]
s7 -> s8 [morphism=edge_1]
s8 -> s9 [morphism=edge_1]
s9 -> __END__
}
\end{lstlisting}

\section{Импорт из aDOT файла}
Тестирование производится путём импорта файла из предыдущего пункта. В случае успешного выполнения получится граф представленный на рисунке (\ref{g2}). Стоит отметить, что внешний вид изменится за счёт применения алгоритма визуализации.

\section{Тестирование циклов в графе}
В описание графовой модели на предыдущем шаге добавим несколько циклов и обратных ребер. Получим следующее описание графовой модели представленное на листинге (\ref{l2}).
\begin{lstlisting}[label=l2, caption={\textit{ Пример описание графовой модели в формате aDOT включающей циклы}}]
digraph TEST
{
// Parallelism
s7 [parallelism=threading]
s11 [parallelism=threading]
s2 [parallelism=threading]
s10 [parallelism=threading]
s1 [parallelism=threading]
// Function
f1 [module=f1_module, entry_func=f1_function]
// Predicates
p1 [module=p1_module, entry_func=p1_function]
// Transition
edge_1 [predicate=p1, function=f1]
// Graph model
__BEGIN__ -> s1
s4 -> s6 [morphism=edge_1]
s5 -> s6 [morphism=edge_1]
s7 => s6 [morphism=edge_1]
s7 => s1 [morphism=edge_1]
s11 => s8 [morphism=edge_1]
s11 => s2 [morphism=edge_1]
s12 -> s8 [morphism=edge_1]
s2 => s4 [morphism=edge_1]
s2 => s5 [morphism=edge_1]
s2 => s7 [morphism=edge_1]
s10 => s11 [morphism=edge_1]
s10 => s12 [morphism=edge_1]
s1 => s2 [morphism=edge_1]
s1 => s10 [morphism=edge_1]
s6 -> s9 [morphism=edge_1]
s8 -> s9 [morphism=edge_1]
s9 -> s6 [morphism=edge_1]
s9 -> __END__
}
\end{lstlisting}

В случае успешной загрузки графа получим модель графа, представленной на рисунке (\ref{g3}).