%----------------------------------------------------------
%Термины и определения по тексту в большинстве случаев выделяются курсивом.
%В настоящем отчете о НИР применяют следующие термины с соответствующими определениями, также используются представленные обозначения и сокращения.
%----------------------------------------------------------
\newglossaryentry{slver}{name={Solver}, description={Решатель системы \gls{dcs-gcd}. Регистрируется в таблице \textbf{com.slvrs} БД \gls{gcddb} \gls{dcs-gcd}.}}

\newabbreviation[category=initialism]{DOT}{DOT}{язык описания графов}
\newabbreviation[category=initialism]{aDOT}{aDOT}{Расширенный формат DOT (\href{https://archrk6.bmstu.ru/index.php/f/777612}{описание представлено в \cite{SokADOT}})}
\newabbreviation[category=initialism]{gbse}{ГПИ}{\href{https://archrk6.bmstu.ru/index.php/f/824891}{графо-ориентированная программная инженерия} (англ., graph-based software engineering (GBSE)), ориентированная для создания программных реализаций \gls{ccm} (патент на изобретение RU 2681408 \cite{patentRU2681408})}

\newabbreviation[category=initialism]{ccm}{СВМ}{Сложный вычислительный метод}


\GlsXtrEnableEntryCounting
{abbreviation}% list of categories to use entry counting
{2}% trigger value

\GlsXtrEnableEntryCounting
{symbol}% list of categories to use entry counting
{2}% trigger value


%----------------------------------------------------------


