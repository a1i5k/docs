%----------------------------------------------------------
\chapter*{ЗАКЛЮЧЕНИЕ}\label{chap_conclusion}
\addcontentsline{toc}{chapter}{ЗАКЛЮЧЕНИЕ}
%----------------------------------------------------------

Была проведена работа по изучению алгоритмов визуализации графов, выделены основные их виды. После просмотра на практике были выделены недостатки и особенности каждого из них, после чего принято решение об использовании алгоритма dot.

Разобраны виды обхода графов, однако из-за специфичности задачи было принято решение о создании собственного алгоритма. Особенностями больше всего влияющие на решение создания алгоритма являются - наличие селекторов, проход не всегда по всему графу.

По результатам тестирования выявлено, что программы выполняет большинство требуемых базовых функций.

Программа сделана с учётом на будущее дополнения каких-либо частей, а именно:
\begin{itemize}
\item Добавления новых атрибутов вершин и дуг графа в файле формате aDot.
\item Использование множества алгоритмов визуализации графов с возможностью выбора определённого.
\item Добавление нескольких алгоритмов обхода графов и возможность выбора какой из них будет использоваться для данного графа.
\end{itemize}
%----------------------------------------------------------
