%----------------------------------------%
% общие определения
\newcommand{\UpperFullOrganisationName}{Министерство науки и высшего образования Российской Федерации}
\newcommand{\ShortOrganisationName}{МГТУ~им.~Н.Э.~Баумана}
\newcommand{\FullOrganisationName}{федеральное государственное бюджетное образовательное\newline учреждение высшего профессионального образования\newline <<Московский государственный технический университет имени Н.Э.~Баумана\newline (национальный исследовательский университет)>> (\ShortOrganisationName)}
\newcommand{\OrganisationAddress}{105005, Россия, Москва, ул.~2-ая Бауманская, д.~5, стр.~1}
%----------------------------------------%
\newcommand{\gitlabdomain}{sa2systems.ru:88}
%----------------------------------------------------------
\newcommand{\doctypesid}{kp} % vkr (выпускная квалификационная работа) / kp (курсовой проект) / kr (курсовая работа) / nirs (научно-исследовательская работа студента) / nkr (научно-квалификационная работа)

% Тема должна быть сформулирована так, чтобы рассказать, о чем работа, но сделать это так, чтобы у читателя возникло желание читать аннота-цию. При формулировке темы не следует стараться рассказать о работе всё. Пример корректной темы: "Математическое моделирование процесса размножения медуз в Южно-Китайском море". Пример некорректной темы: "Применение модели SIS для моделирования процесса размножения медуз в Южно-Китайском море с использованием метода Рунге-Кутты и многопроцессорных вычислительных систем".
\newcommand{\Title}{Разработка web-приложения, обеспечивающего построение графовых моделей алгоритмов обработки данных}%{}
\newcommand{\TitleSource}{кафедра} % кафедра, предприятие, НИР, НИР кафедры, заказ организации

\newcommand{\SubTitle}{по дисциплине <<Модели и методы анализа проектных решений>>} % Методы оптимизации
\newcommand{\faculty}{<<Робототехника и комплексная автоматизация>>}
\newcommand{\facultyShort}{РК}
\newcommand{\department}{<<Системы автоматизированного проектирования (РК-6)>>}
\newcommand{\departmentShort}{РК-6}

\newcommand{\Author}{Журавлев~Н.В.}
\newcommand{\AuthorFull}{Журавлев~Николай~Вадимович}
\newcommand{\ScientificAdviserPosition}{доцент, к.ф.-м.н.}	% Должность научного руководителя
\newcommand{\ScientificAdviser}{Соколов~А.П.}	% Научный руководитель
%\newcommand{\ConsultantA}{@Фамилия~И.О.@}				% Консультант 1
%\newcommand{\ConsultantB}{@Фамилия~И.О.@}				% Консультант 2
\newcommand{\Normocontroller}{Грошев~С.В.}		% Нормоконтролёр
\newcommand{\group}{РК6-72Б}
\newcommand{\Semestr}{осенний семестр} % Например: осенний семестр или весенний семестр
\newcommand{\BeginYear}{2022}
\newcommand{\Year}{2023}
\newcommand{\Country}{Россия}
\newcommand{\City}{Москва}
\newcommand{\TaskStatementDate}{<<\underline{\textit{10}}>> \underline{октября} \Year~г.} %Дата выдачи задания

\newcommand{\depHeadPosition}{Заведующий кафедрой}		% Должность руководителя подразделения
\newcommand{\depHeadName}{А.П.~Карпенко}		% Должность руководителя подразделения

% Цель выполнения
\newcommand{\GoalOfResearch}{создать основу для построения/сохранения/накопления графовый моделей, описывающих тот или иной процесс обработки данных} % с маленькой буквы и без точки на конце

% Объектом исследования называют то, что исследуется в работе. Например, для указанной выше темы объектом может быть популяция медуз, но никак ни модель SIS, ни Южно-Китайское море, ни метод моделирования популяции медуз.
\newcommand{\ObjectOfResearch}{архитектура процессов обработки данных}

% Предмет исследований (уже чем объект, определяется, исходя из задач: формулируется как существительное, как правило, во множественном числе, определяющее "конкретный объект исследований" среди прочих в рамках более общего)
\newcommand{\SubjectOfResearch}{визуализатор графовых моделей}

% Основная задача, на решение которой направлена работа
\newcommand{\MainProblemOfResearch}{создание визуализатора графовых моделей, который должен описывать объект проектирования, который имеет какие-либо параметры, изменяемые в процессе обхода графа.}

% Выполненные задачи
\newcommand{\SubtasksPerformed}{%
В результате выполнения работы:
\begin{inparaenum}[1)]
	\item предложено создание визуализатора графовых моделей;
	\item создан алгоритм обхода ориентированного графа;
	\item разработано алгоритмы визуализации и обхода графа. Возможность чтения и вывод из файла формата aDot;
\end{inparaenum}}

% Ключевые слова (представляются для обеспечения потенциальной возможности индексации документа)
\newcommand{\keywordsru}{%
	ориентированные графы, обход графа, визуализация графа, aDot, web-приложение} % 5-15 слов или выражений на русском языке, для разделения СЛЕДУЕТ ИСПОЛЬЗОВАТЬ ЗАПЯТЫЕ
\newcommand{\keywordsen}{%
	@keywordsen@} % 5-15 слов или выражений на английском языке, для разделения СЛЕДУЕТ ИСПОЛЬЗОВАТЬ ЗАПЯТЫЕ

% Краткая аннотация
\newcommand{\Preface}{
Работа посвящена разработке визуализатора графовых моделей, с некоторыми особенностями. А именно, граф должен описывать объект проектирования, который имеет какие-либо параметры, которые должны изменяться в графе. Узлами графа является определенные значения параметров изучаемого объекта. В ребрах должен исполняться код на языке Python, предварительно занесённый туда пользователем. Этот код должен изменять какие-либо параметры объекта проектирования. При разработке для визуализации графа был использован алгоритм dot, а для обхода графа был создан собственный.
} % с большой буквы с точкой в конце

%----------------------------------------%
% выходные данные по документу
\newcommand{\DocOutReference}{\Author. \Title\xspace\SubTitle. [Электронный ресурс] --- \City: \Year. --- \total{page} с. URL:~\url{https://\gitlabdomain} (система контроля версий кафедры РК6)}

%----------------------------------------------------------

